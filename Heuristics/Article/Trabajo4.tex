\documentclass[10pt, twoside]{article}

\usepackage[utf8]{inputenc}
\usepackage{natbib}
\usepackage{graphicx}
\usepackage{booktabs}
\usepackage{adjustbox}
\usepackage[table,xcdraw]{xcolor}
\usepackage{amsmath}
\usepackage[]{todonotes}
\usepackage{fancyhdr}
\usepackage{multirow}
\usepackage{float}
\usepackage[spanish,onelanguage]{algorithm2e}
\usepackage{caption}
\captionsetup[table]{skip=10pt}

\usepackage[spanish, es-tabla]{babel}

\title{Problema del flujo de trabajos}
\author{\emph{Alejandro Salgado Gómez} y \emph{Juan Carlos Rivera}\\
\vspace{0.3cm}
\small{\tt{asalgad2@eafit.edu.co, jrivera6@eafit.edu.co}}\\
Departamento de Ciencias Matemáticas\\
Escuela de Ciencias\\
Universidad EAFIT\\
Medellín -- Colombia}
\date{}

\usepackage{anysize}
\marginsize{3cm}{2cm}{2cm}{3cm}

\pagestyle{fancy}
\lfoot[\date{\today}]{\date{\today}}
\rfoot[\thepage]{\thepage}
\cfoot[]{}
\rhead[Rivera et al. (2017)]{Problema del flujo de trabajos}
\lhead[]{}

\fancypagestyle{firststyle}
{
   \fancyhf{}
   \fancyfoot[R]{Última actualización: Julio de 2017}
}

\sloppy

\begin{document}

    \maketitle

    \thispagestyle{firststyle}

\begin{abstract}
En este trabajo se propenen diversos algoritmos para encontrar
soluciones a problemas de flujo de trabajos. Dichos métodos fueron ejecutados
con diferentes instancias de problemas encontradas en la literatura, con el
objetivo de comparar su desempeño y conocer sus fortalezas y debilidades a
la hora de resolver este tipo de problemas. Se encontró que las diferencias en
los tiempos de ejecución que estos métodos presentan son muy notorias,
especialmente cuando el número de trabajos es incrementado a 100. También
se encontró que existe una marcada distinción entre la calidad de las
soluciones que estos algoritmos pueden llegar a producir.\\

\noindent \textbf{Palabras Clave:}
Heurística, Optimización, Modelación, Flujo de trabajos.
\end{abstract}

\section{Introducción}

El problema de flujo de trabajos ha atraído la atención de un gran número de
investigadores del área de optimización desde que fue propuesto por primera vez
en 1954, y a impulsado la implementación de un número importante de
aproximaciones que buscan dar una solución a este problema. Desde su primera
formulación han surgido diversas variaciones del problema, por ejemplo, se puede
permitir que los trabajos tomen un orden distinto en cada una de las máquinas,
generando un total de $(n!)^m$ posibles combinaciones. Sin embargo, en este
trabajo se utilizará la versión en la que los trabajos se encuentran en una linea
de producción y por lo tanto su orden no puede ser variado de una máquina a otra,
reduciendo las posibles combinaciones a $n!$. Finalmente, aunque a primera
vista parece un problema sencillo de resolver después de realizar esta
simplificación, en \cite{complex} se demostró que es un problema
NP-Hard.\\

Se han utilizado diferentes aproximaciones para resolver este problema, tal como
estrategias exactas, pero también de métodos heurísiticos, puesto que las
instancias de este problema pueden llegar a representar una alta complejidad
cuando se trabaja con una cantidad considerable de trabajos. Esto debido a su
crecimiento factorial en cuanto al número de secuencias posibles que se deben
considerar. En \cite{heuristico4}, \cite{heuristico1}, \cite{heuristico2} y
\cite{heuristico3} se pueden encontrar diferentes propuestas para usar una
variedad de métodos heurísticos que buscan encontrar una solución eficiente
para esta y otras posibles variaciones del problema.\\

Esta trabajo está organizado de la siguiente manera, en primer lugar
se da una explicación del problema en cuestion y se describirán algunas
características importantes, posteriormente se especificará el modelo
matemático del problema, luego se procederá a detallar la implementación de los
algoritmos utilizados para encontrar las soluciones y su respectiva ejecución,
por último se detallan algunas conclusiones.\\

\section{Descripción del problema y formulación matemática}\label{sec_mathmod}

El problema de flujo de trabajos fue formulado por primera vez por \cite{original}
como un problema compuesto por 2 elementos principales, el primero es una serie
de trabajos, estos, debido a su gran parecido tienen un proceso que puede ser
asumido como igual, y el segundo componente es conjunto de máquinas ordenadas,
por las cuales los trabajos deben pasar para poder ser completados. En el problema,
los trabajos deben ser procesados por una línea de producción, lo que significa
que la secuencia de trabajos no puede ser cambiada durante el proceso.
Finalmente, el objetivo consiste en buscar una secuencia de trabajos que
minimice alguna medida de costo productivo, tal como puede ser el tiempo total
de producción, máximo tiempo para completar un trabajo, entre otros.
\\

Es importante mencionar que en la propuesta original del problema la cantidad
de máquinas estaba limitada a un total de 3, sin embargo, la cantidad de trabajos
podría ser un número $n$ arbitrario. Esta idea luego se extendió a un problema
donde se tuvieran $n$ trabajos y $m$ máquinas, esto con el objetivo de tener una
descripción más general que pudiera ser utilizada en una variedad
más grande de aplicaciones. Un modelo de programación lineal entera para el tipo
de problema original puede ser encontrado en \cite{old_model}, sin embargo, en
\cite{model} se puede encontrar un modelo matemático que se puede utilizar para
resolver el problema generalizado. El siguiente es una descripción de dicho
modelo generalizado.\\

\noindent
\textbf{Indices}\\
$i$: Representa trabajos (1...$n$)\\
$j$: Representa la posición en la secuencia (1...$n$)\\
$k$: Representa máquinas (1...$m$)\\

\noindent
\textbf{Variables}\\
$z_{ij}$: 1 si el trabajo $i$ es asignado en la posición $j$, 0 en otro caso\\
$x_{jk}$: Tiempo inactivo de la máquina $k$ por esperar el trabajo $j$\\
$y_{jk}$: Tiempo inactivo del trabajo $j$ mientras que espera a la máquina $k+1$\\
$f$: Tiempo total consumido para procesar los trabajos\\

\noindent
\textbf{Datos}\\
$t_{ki}$: Tiempo de procesamiento en la máquina $k$ para el trabajo $i$\\

\noindent
\textbf{Función objetivo}
\begin{center}
$\min f$
\end{center}

\noindent
\textbf{Restricciones}

\begin{align}
    \label{eq_1} \sum_{j=1}^{n} z_{ij} &= 1 \quad i=1...n \\
    \label{eq_2} \sum_{i=1}^{n} z_{ij} &= 1 \quad j=1...n \\
    \label{eq_3} \sum_{l=1}^{n}t_{kl} * z_{l,j+1} + y_{j+1,k} + x_{j+1,k} = \sum_{l=1}^{n} t_{k+1,l} * z_{lj} + &y_{jk} + x_{j+1,k+1} \quad j=1...n-1 \quad k=1...m-1\\
    \label{eq_4} f = \sum_{j=1}^{n} \sum_{l=1}^{n} t_{ml} &* z_{lj} + \sum_{j=1}^{n} x_{jm}\\
    \label{eq_5} y_{ik} = 0& \quad k=1...m-1
\end{align}
\begin{align}
    \label{eq_6} x_{ik} = z_1 &\sum_{p=1}^{k-1} t_p \quad k=2...m
\end{align}

En este modelo las restricciones \ref{eq_1} y \ref{eq_2} establecen las condiciones
de un problema de asignación convencional, en el que a cada trabajo debe ser
asignado a una posición y a su vez cada una de las posiciones tiene que tener
asignado solo un trabajo específico. Por otro lado, la restricción número \ref{eq_3}
establece las reglas para determinar si una serie de tiempo es valida y puede
ser utilizada como una solución factible para el problema. Luego, la restricción
\ref{eq_4} establece como se debe calcular el valor que será utilizado
como función objetivo, la cual en este caso es tomada como el tiempo consumido para
procesar los trabajos, sin embargo, en \cite{model} también se puede encontrar
una variedad de otro tipo de funciones que pueden ser utilizadas para optimizar
diferentes partes del proceso. Finalmente las restricciones \ref{eq_5} y \ref{eq_6}
son utilizadas para que los valores de las variables que describen los tiempos
de espera relacionados con el primer trabajo de la secuencia tengan una
inicialización correcta.\\

\section{Algoritmos de solución}

Para buscar las soluciones del problema se utilizaron 5 aproximaciones, un
método constructivo de mejor inserción, el cual fue utilizado con y sin
ruido en los datos de entrada, una construcción GRASP, un algoritmo de busqueda
local y uno de vecindarios variables.\\

Todas las implementaciones utilizan el algoritmo \ref{algorithm1} para calcular
los tiempos de finalización de un trabajo en cada una de las máquinas teniendo
en cuenta las restricciones de ocupación, la cual es una de las partes
fundamentales del cómputo.\\

\RestyleAlgo{boxruled}
\begin{algorithm}[H]
    \For{ cada máquina que falte por ser utilizada }{
        fin\_uso = max(fin\_uso\_anterior, fin\_uso\_predecesor) + tiempo\_uso\;
    }
    \caption{Cálculo de tiempos de terminación para un trabajo en cada máquina}
    \label{algorithm1}
\end{algorithm}

\vspace{0.5cm}

Ahora, el primer método, es decir, el algoritmo de mejor inserción, encuentra
una solución al buscar cual sería la mejor posición para insertar un trabajo,
siempre tatando de mantener el incremento de tiempo lo menor posible. Este
procedimiento, toma como elemento inicial a aquel trabajo que presente un tiempo de
procesamiento más corto, luego calcula los valores de la función objetivo
al insertar dicho elemento en todas las posiciones posibles, con la intención
de quedarse con la mejor opción. Finalmente, agrega dicho trabajo a la
secuencia y repite el proceso hasta que todos los trabajos han sido
asignados.\\

Por otro lado, esta primera aproximación puede descartar combinaciones de
trabajos que podrían tener un costo menor, debido a la secuencia estricta que el
algoritmo va construyendo. Por este motivo, en el segundo método, se cambian los
tiempos de procesasamiento de los trabajos con la ayuda una distribución
uniforme centrada en 0, y se vuelve a encontrar una solución por medio de la
mejor inserción. Esto con el objetivo de permitir que esta nueva construcción
acepte distintas combinaciones de trabajos y pueda analizar un espacio de la
región factible más diverso, generando la posibilidad de llegar a un mejor
resultado.\\

También se utilizó una construcción GRASP, este método consiste en la
construcción de una lista de candidatos, los cuales son un subconjunto
de trabajos que representan las mejores elecciones posibles que se pueden hacer
con la solución que se ha construido hasta el momento. Luego, se selecciona un
trabajo de la lista aleatoriamente y se utiliza como el siguiente elemento. Por
ultimo, se vuelve a calcular una lista de candidatos con las opciones restantes
y se repite el procedimiento hasta obtener una solución completa.\\

Adicionalmente se implementó el algoritmo Ils tomando como referencia la
experiencia de \cite{ils}, en la que se obtuvieron buenos resultados con esta
aproximación. Este método consiste en realizar una busqueda local iterativa en
la que se genera una perturbación a la solución en cada una de las iteraciones.
Para este caso el vecindario de la busqueda fue definido tomando como referencia
a \cite{localsearch}, donde se elimina un elemento de la solución y se intenta
buscar una nueva posición que mejore la función objetivo. En segundo lugar, el
método de perturbación escogido consiste en intercambiar dos elementos de la solución
aleatoriamente. Por ultimo, el algoritmo se ejecuta hasta que se llega a un
número de iteraciones definido al inicio del programa.\\

Para el último método se implementó un algoritmo de busqueda por vecindarios. Este
algoritmo consiste en buscar la mejor solución posible al ejecutar busquedas
locales sobre distintos vecindarios, con el objetivo de solucionar el problema
de los optimos locales al utilizar diversas tecnicas de busqueda. Para este método
se utlizaron dos vecindarios, el primero fue también utilizado en la aproximación
descrita anteriormente, el segundo es una modificación de ese vecindario en donde
no solo se elimina uno de los trabajos de la secuencia, sino que se extraen dos
trabajos y luego se busca la mejor inserción considerando uno después del otro.

\cite{genetic1}
\cite{genetic2}
\cite{genetic3}
% Order preserving one-point crossover
% seleccion por ruleta

\section{Experimentación computacional}

Se utilizaron los algoritmos anteriores para resolver los problemas
propuestos en \cite{dataset}. Para crear las instancias de dicho articulo se desarrollo
un algoritmo en C y se implementó un programa en Python para comprobar que dichas
instancias fueran correctas. En las tablas \ref{table:1}, \ref{table:2} y
\ref{table:3} se comparan diferentes criterios de los resultados obtenidos con
cada uno de los algoritmos propuestos con las mejores soluciones encontradas en
\cite{literature}, donde se recogieron las mejores soluciones de la literatura para este
conjunto de instancias. Vale la pena aclarar que en la tabla \ref{table:2} se
muestra el error porcentual que se obtuvo con cada uno de los algoritmos al tomar la mejor
solución de la literatura como el valor óptimo. También es importante mencionar
que en el caso de los métodos que contienen un componente aleatorio, se repitió
cada ejecución cinco veces y se escogio la mejor respuesta encontrada.
Finalmente, se encontró que uno de los algoritmos requiere de un tiempo de
ejecución demasiado alto, y debido a esto no fue posible encontrar una
respuesta a algunas instancias, para estos casos se utiliza la notación N/A.

    \begin{table}[H]
        \centering
        \caption{Comparaciones de la calidad de las soluciones}
        \resizebox{\textwidth}{!}{
        \begin{tabular}{|c|c|c|c|c|c|c|c|c|c|c|c|c|c|c|c|}
            \hline
Instancia &     Inserción   &  \multicolumn{3}{c|}{Construcción GRASP}   &    \multicolumn{3}{c|}{Ruido}  &  \multicolumn{3}{c|}{Ils}  &  Vnd   & \multicolumn{3}{c|}{Genético} & Literatura \\
\hline
          &                 & $\alpha$=0.2 & $\alpha$=0.5 & $\alpha$=0.8 &   $r$=20   &    $r$=50   &   $r$=80    & $n$=50 & $n$=100 & $n$=150 &        & $p$=50 & $p$=100 & $p$=150 &     \\
\hline
20x5      &      14355      &    15149     &     16181    &     17124    &   14327    &    15670    &   15833     & 14098  & 14055   &  14033  & 14041  & 14403  & 14194   & 14197   & 14033 \\
\hline
20x10     &      21285      &    24803     &     24256    &     25486    &   22228    &    23118    &   23828     & 21124  & 21058   &  21112  & 21187  & 21207  & 21285   & 21004   & 21187 \\
\hline
20x20     &      34564      &    36087     &     36607    &     38019    &   34859    &    36494    &   36437     & 34005  & 34005   &  34005  & 34050  & 34396  & 34011   & 33961   & 33623 \\
\hline
50x5      &      67361      &    82527     &     82046    &     80865    &   69116    &    74622    &   79553     & 66704  & 66258   &  66375  & 66266  & 70123  & 69502   & 68930   & 64980 \\
\hline
50x10     &      91510      &    107244    &     107457   &     110893   &   94927    &    101266   &   106719    & 90882  & 90318   &  90372  & 90520  & 95003  & 93287   & 92970   & 87979 \\
\hline
50x20     &      131306     &    149801    &     149917   &     150322   &   134516   &    141701   &   143894    & 129679 & 129046  &  129270 & 128645 & 135302 & 133689  & 132261  & 126338\\
\hline
100x5     &      262371     &    309234    &     322253   &     321560   &   270820   &    291362   &   296366    & 261260 & 260983  &  260111 & 258253 & 283687 & 278795  & 276355  & 256061\\
\hline
100x10    &      312914     &    378962    &     375931   &     381073   &   320916   &    344729   &   349935    & 311261 & 308779  &  310942 & 307438 & 330807 & 332547  & 325170  & 301453\\
\hline
100x20    &      390176     &    447530    &     450123   &     452952   &   398672   &    420575   &   432199    & 385060 & 381777  &  380227 & 381902 & 407432 & 400144  & 400947  & 370035\\
\hline
        \end{tabular}}
        \label{table:1}
    \end{table}
    \begin{table}[H]
        \centering
        \caption{Comparaciones relativas de calidad de las soluciones}
        \resizebox{\textwidth}{!}{
        \begin{tabular}{|c|c|c|c|c|c|c|c|c|c|c|c|c|c|c|}
            \hline
            Instancia & Mejor inserción &  \multicolumn{3}{c|}{Construcción GRASP}   &     \multicolumn{3}{c|}{Ruido}    &  \multicolumn{3}{c|}{Ils}  &  Vnd & \multicolumn{3}{c|}{Genético} \\
            \hline
                      &                 & $\alpha$=0.2 & $\alpha$=0.5 & $\alpha$=0.8 &  $r$=20  &    $r$=50  &   $r$=80  & $n$=50 & $n$=100 & $n$=150 &      & $p$=50 & $p$=100 & $p$=150 \\
            \hline
            20x5      &      2.29       &    7.95      &     15.31    &     22.03    &   2.10   &    11.67   &    12.83  &  0.46  &  0.16   &   0.00  & 0.06 &   2.64 &   1.15  &  1.17 \\
            \hline
            20x10     &      0.46       &    17.07     &     14.49    &     20.29    &   4.91   &    9.11    &    12.47  & -0.30  & -0.61   &  -0.35  & 0.00 &   0.09 &   0.46  & -0.86 \\
            \hline
            20x20     &      2.80       &    7.33      &     8.87     &     13.07    &   3.68   &    8.54    &    8.37   &  1.14  &  1.14   &   1.14  & 1.27 &   2.30 &   1.15  &  1.01 \\
            \hline
            50x5      &      3.66       &    27.00     &     26.26    &     24.45    &   6.37   &    14.84   &    22.43  &  2.65  &  1.97   &   2.15  & 1.98 &   7.91 &   6.96  &  6.08 \\
            \hline
            50x10     &      4.01       &    21.90     &     22.14    &     26.04    &   7.90   &    15.10   &    21.30  &  3.30  &  2.66   &   2.72  & 2.89 &   7.98 &   6.03  &  5.67 \\
            \hline
            50x20     &      3.93       &    18.57     &     18.66    &     18.98    &   6.47   &    12.16   &    13.90  &  2.64  &  2.14   &   2.32  & 1.83 &   7.10 &   5.82  &  4.69 \\
            \hline
            100x5     &      2.46       &    20.77     &     25.85    &     25.58    &   5.76   &    13.79   &    15.74  &  2.03  &  1.92   &   1.58  & 0.86 &  10.79 &   8.88  &  7.93 \\
            \hline
            100x10    &      3.80       &    25.71     &     24.71    &     26.41    &   6.46   &    14.36   &    16.08  &  3.25  &  2.43   &   3.15  & 1.99 &   9.74 &  10.31  &  7.87 \\
            \hline
            100x20    &      5.44       &    20.94     &     21.64    &     22.41    &   7.74   &    13.66   &    16.80  &  4.06  &  3.17   &   2.75  & 3.21 &  10.11 &  8.14   &  8.35 \\
            \hline
        \end{tabular}}
        \label{table:2}
    \end{table}

    \begin{table}[H]
        \centering
        \caption{Comparaciones de tiempo}
        \resizebox{\textwidth}{!}{
        \begin{tabular}{|c|c|c|c|c|c|c|c|c|c|c|c|c|c|c|}
            \hline
            Instancia & Mejor inserción &  \multicolumn{3}{c|}{Construcción GRASP}   &    \multicolumn{3}{c|}{Ruido}   & \multicolumn{3}{c|}{Ils}   &   Vnd  & \multicolumn{3}{c|}{Genético} \\
            \hline
                      &                 & $\alpha$=0.2 & $\alpha$=0.5 & $\alpha$=0.8 &  $r$=20  &   $r$=50  &  $r$=80  & $n$=50 & $n$=100 & $n$=150 &        & $p$=50 & $p$=100 & $p$=150 \\
            \hline
            20x5      &      0.11       &  $<$ 0.01    &   $<$ 0.01   &   $<$ 0.01   &   0.08   &    0.08   &  0.08    &  2.17  &    4.44 &    6.47 & 0.59   &  1.70  &  3.79   &  6.18 \\
            \hline
            20x10     &      0.15       &  $<$ 0.01    &   $<$ 0.01   &   $<$ 0.01   &   0.11   &    0.11   &  0.11    &  3.38  &    6.03 &    8.97 & 0.32   &  2.09  &  4.56   &  7.51 \\
            \hline
            20x20     &      0.25       &  $<$ 0.01    &   $<$ 0.01   &   $<$ 0.01   &   0.17   &    0.17   &  0.17    &  5.54  &    9.78 &   14.53 & 0.62   &  2.83  &  6.82   & 10.82 \\
            \hline
            50x5      &      2.97       &    0.01      &     0.01     &     0.01     &   2.49   &    2.23   &  2.31    &  33.19 &   61.24 &   90.85 & 15.79  &  3.24  &  6.72   & 10.69 \\
            \hline
            50x10     &      4.39       &    0.02      &     0.02     &     0.02     &   3.27   &    3.02   &  3.01    &  49.59 &   92.00 &  140.67 & 14.32  &  4.17  &  8.66   & 14.47 \\
            \hline
            50x20     &      7.41       &    0.03      &     0.03     &     0.03     &   4.56   &    4.72   &  4.86    &  80.53 &  153.37 &  225.64 & 30.68  &  6.31  & 14.60   & 20.61 \\
            \hline
            100x5     &      41.93      &    0.05      &     0.05     &     0.05     &   31.99  &    36.99  &  36.96   & 268.95 &  509.40 &  726.47 & 389.86 &  5.17  & 11.43   & 17.76 \\
            \hline
            100x10    &      63.69      &    0.06      &     0.07     &     0.07     &   50.91  &    50.26  &  51.97   & 407.07 &  760.98 & 1191.70 & 287.35 &  7.53  & 15.25   & 23.89 \\
            \hline
            100x20    &      107.56     &    0.10      &     0.10     &     0.10     &   63.90  &    63.78  &  64.36   & 687.57 & 1286.21 & 1876.94 & 309.12 & 11.84  & 23.37   & 35.95 \\
            \hline
        \end{tabular}}
        \label{table:3}
    \end{table}

\section{Conclusiones}

En este trabajo se realizó una descripción del problema de flujo de trabajos
clásico y se encontró que dicho problema pertenece a la categoría de problemas
NP-Hard. Debido a su complejidad se implementaron cinco métodos heurísticos
para encontrar una solución a distintas instancias del problema. En cuanto a
los experimentos realizados, se puedo observar que el método de construcción
GRASP no es muy apropiado para este tipo de problema, ya que presento los
peores rendimientos en cuanto a calidad, aunque con un tiempo de ejecución muy
bajo.  Esto sugiere que un método constructivo del tipo vecino más cercano, así
como una construcción aleatoria, no son buenas aproximaciónes para resolver el
problema en cuestión.  Por otro lado, se encontró que los algoritmos de mejor
inserción y ruido resultan en soluciones aceptables que se pueden encontrar con
un tiempo de ejecución excelente. En contraste con estas aproximaciones, se
tienen los métodos de busqueda local y por vecindarios, los cuales resultan en
soluciones de excelente calidad, llegando incluso al óptimo en un caso, pero
requieren una cantidad considerable de tiempo para ser ejecutados a medida que
el tamaño del problema va aumentando. Esto se ve especialmente reflejado en el
caso del algoritmo Vnd, donde no se pudo encontrar una solución a los casos más
grandes debido a que excedió la cantidad de tiempo de cómputo disponible.

{\small
\bibliographystyle{authordate1}

\bibliography{Trabajo4}}

\end{document}
